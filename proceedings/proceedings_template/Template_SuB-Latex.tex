%!TEX TS-program = pdflatex
\documentclass[a4paper,12pt,twoside]{article}



\usepackage[T1]{fontenc}
\usepackage{tgtermes}
\usepackage{newtxmath}          
\linespread{0.97}


\makeatletter
\newcommand*\MySupRaise{1.05ex}
\newcommand*\MySupSize{\scriptsize}

\DeclareRobustCommand{\textsuperscript}[1]{%
  \leavevmode\hbox{\raise\MySupRaise\hbox{{\normalfont\MySupSize #1}}}%
}

\renewcommand\@makefnmark{\hbox{\textsuperscript{\@thefnmark}}}
\makeatother



\usepackage{parskip}
\usepackage[top=1.5in,left=1in,right=1in,bottom=1in]{geometry}
%\setlength{\parindent}{0pt}
%\setlength{\parskip}{\the\baselineskip}
\pagestyle{empty} 



\makeatletter
\renewcommand\@makefntext[1]{%
  \parindent 1em\noindent
  \hbox{\@makefnmark}#1}
\renewcommand\large{\@setfontsize\large{14pt}{18}}
\makeatother

\usepackage{titlesec}
\titleformat{\subsection}{\normalfont\normalsize}{\thesubsection.}{1ex}{}
\titleformat{\subsubsection}{\normalfont\normalsize}{\thesubsubsection.}{1ex}{}
\titleformat{\section}{\normalfont\normalsize\bfseries}{\thesection.}{.63cm}{}
\titlespacing*{\section}{0pt}{24pt}{12pt}
\titlespacing*{\subsection}{0pt}{24pt}{12pt}
\titlespacing*{\subsubsection}{0pt}{2ex plus 1ex minus .2ex}{2.3ex plus .2ex}
\setlength{\parindent}{0pt}
\setlength{\parskip}{2.5ex plus 0.5ex minus 0.2ex}




\usepackage{natbib} 
\bibpunct{(}{)}{;}{a}{,}{,}


\setlength{\bibsep}{0pt} \relax
\setcitestyle{notesep={: },yysep={, }} \relax

\usepackage{linguex,url}
\def\refdash{} % To suppress the dash in (2-a) etc. Defining both as different versions of linguex had different names for this command.
\def\firstrefdash{}


\usepackage{titleps}
\usepackage{lipsum}
\usepackage{xcolor}
\newpagestyle{mystyle}{
\sethead[\thepage][\Author][]{}{\Title}{\thepage}
}
\pagestyle{mystyle}
\author{Quartz Colvin}
\title{A cover analysis of the plural classifier in Hmong}
\makeatletter
\newcommand\Author{First Author -- Second Author}
\let\Title\@title
\makeatother
%header with author and title

\pagenumbering{gobble} 
%suppresses page numbering in header



\begin{document}
%%\maketitle
\setlength{\Extopsep}{0pt}
\thispagestyle{empty}

{\large \textbf{A cover analysis of the plural classifier in Hmong}}\footnote{We would like to thank my consultants: KX and YX. I also want to thank my Qualifying Paper committee (Dr. Dorothy Ahn, Dr. Yimei Xiang, and Dr. Maria Kouneli) for their guidance with this project.}\\
Quartz COLVIN --- \textit{Rutgers University}\\

\textbf{Abstract.} \textcolor{red}{???fillin}

\textbf{Keywords:} \textcolor{red}{key, words, hello.???}

\section{Introduction}

In this paper, I show that Hmong \textit{cov} is a plural classifier that generates a set of subsets (a cover). Classifiers in Hmong (Hmong-Mien) typically give rise to definite readings via an $\iota$ operator in the absence of an indefinite article. The plural, group classifier cov generates a cover that can be seen clearly in definite and indefinite contexts.


\subsection{Subsection} 

This is what a subsection looks like. Here is an example, namely example \ref{Example1}.  I made it with \texttt{linguex.sty}.  If you want to make your examples with \texttt{expex} or \texttt{gb4e} or whatever, you can.

\ex. \label{Example1}I am an example \hfill (Hello!)\\ Maybe I need two lines.

\ex. Me, too.
	\a. \label{AnotherRef}I am \ref{AnotherRef}.
	\b. I am not.

\exg. De~temps~en~temps je m'ennuie de l'anglais.\\
occasionally I \textsc{me}.bore of the.English\\
`Occasionally I get bored of English.'

\ex.	
\ag. Maintenant, par exemple.\\
	now by example\\
	`Now, for example.'
\bg. I ara tamb\'{e}.\\
and now also \\
`And now, as well.'

Here are some examples of references in the main text: \citet{Montague73, HeimKratzer98}.  I could also cite them parenthetically \citep{Montague73, HeimKratzer98}.  I did this with \texttt{natbib.sty}.  Please use \texttt{natbib.sty}. Here are some examples of references in the main text: \citet{Dowty78}, \citet{HeimKratzer98}.  If you want to cite \citeauthor{kz84}'s \citeyearpar{kz84} paper possessively you can. Please note that \texttt{natbib.sty} truncates ``to appear'' \citep[as in][]{Montague73a} in citations; you must use a hack if you want to cite \nocite{Montague73a} Montague (to appear). 

I could also cite them parenthetically Here are some examples of references in the main text: \citet{Montague73, HeimKratzer98}. 




\bibliographystyle{chicago}
\bibliography{SuB}

\newpage
\subsection{Towards the end}

For other tasks, you can use pretty much any other package you like, so long as you don't alter the margins, spacing between paragraphs, font, font size, etc.


\newpage

Lorem ipsum dolor sit amet, consetetur sadipscing elitr, sed diam nonumy eirmod tempor invidunt ut labore et dolore magna aliquyam erat, sed diam voluptua. At vero eos et accusam et justo duo dolores et ea rebum. Stet clita kasd gubergren, no sea takimata sanctus est Lorem ipsum dolor sit amet. Lorem ipsum dolor sit amet, consetetur sadipscing elitr, sed diam nonumy eirmod tempor invidunt ut labore et dolore magna aliquyam erat, sed diam voluptua. At vero eos et accusam et justo duo dolores et ea rebum. Stet clita kasd gubergren, no sea takimata sanctus est Lorem ipsum dolor sit amet.

\end{document}
